
\subsection{Exercice 2-1}
\subsubsection{Enoncé}
Une tasse de café à $60\celsius$ de 1dl dont la chaleur massique est de $4180[\cm]$. On y rajoute $12[g]$ de crème à $4\celsius$ dont la chaleur massique est de $3350[\cm]$.

\textit{A quelle température allez-vous boire le café ?}
\subsubsection{Valeurs}
\begin{itemize}
    \item[crème] (1)
    \item[] 
    \begin{itemize}
        \item[$T1$] = $4\celsius$
        \item[$m1$] = $12[g]$ = $0.012[kg]$
        \item[$Cm1$] = $3350[\cm]$
    \end{itemize} 
    \item[café] (2)
    \item[] 
    \begin{itemize}
        \item[$T2$] = $60\celsius$
        \item[$m2$] = $1[dl]$ = $0.1[kg]$
        \item[$Cm2$] = $4180[\cm]$   
    \end{itemize} 
\end{itemize}
\subsubsection{Reponse}
\begin{itemize}
    \item[Calcul des Q] : $Q = Eth = m * Cm * (Tf-Ti)$
    \item $Q1$ = $0.012 * 3350 * (Tf - 4) = 40.2 * (Tf-4) = 40.2Tf - 160.80$
    \item $Q2$ = $0.1 * 4180 * (Tf - 60)$ = $418 * (Tf - 60)$ = $418Tf - 25080$
    \item[Conservation des Q] : $Q1 + Q2 = 0$
    \item $Q1 + Q2 = 458.2Tf - 24919.2 = 0$
    \item $Tf$ = $\frac{24919.2}{458.2} = 55\celsius $
\end{itemize}

\subsection{Exercice 2-2}
\subsubsection{Enoncé}
Une tasse de café à $60\celsius$ de 1dl dont la chaleur massique est de $4180[\cm]$. On y rajoute $12[g]$ de crème à $4\celsius$ dont la chaleur massique est de $3350[\cm]$. On rajoute une tasse en pyrex dont la chaleur massiques est de $830[\cm]$.

\textit{A quelle température allez-vous boire le café ?}
\subsubsection{Valeurs}
\begin{itemize}
    \item[crème] (1)
    \item[] 
    \begin{itemize}
        \item[$T1$] = $4\celsius$
        \item[$m1$] = $12[g]$ = $0.012[kg]$
        \item[$Cm1$] = $3350[\cm]$
    \end{itemize} 
    \item[café] (2)
    \item[] 
    \begin{itemize}
        \item[$T2$] = $60\celsius$
        \item[$m2$] = $1[dl]$ = $0.1[kg]$
        \item[$Cm2$] = $4180[\cm]$   
    \end{itemize} 
    \item[tasse en pyrex] (3)
    \item[] 
    \begin{itemize}
        \item[$T3$] = $20\celsius$
        \item[$m3$] = $0.1[kg]$
        \item[$Cm3$] = $830[\cm]$   
    \end{itemize} 
\end{itemize}
\subsubsection{Reponse}
\begin{itemize}
    \item[Calcul des Q] : $Q = Eth = m * Cm * (Tf-Ti)$
    \item $Q1$ = $0.012 * 3350 * (Tf - 4) = 40.2 * (Tf-4) = 40.2Tf - 160.80$
    \item $Q2$ = $0.1 * 4180 * (Tf - 60)$ = $418 * (Tf - 60)$ = $418Tf - 25080$
    \item $Q3$ = $0.1 * 830 * (Tf - 20)$ = $83 * (Tf - 20)$ = $83Tf - 1660$
    \item[Conservation des Q] : $Q1 + Q2 +Q3 = 0$
    \item $Q1 + Q2 + Q3 = 541.2 - 26579.2 = 0$
    \item $Tf$ = $\frac{26579.2}{541.2} = 49.1\celsius $
\end{itemize}


\subsection{Exercice 2-3}
\subsubsection{Enoncé}
Une tasse de café à $60\celsius$ de 1dl dont la chaleur massique est de $4180[\cm]$. On y rajoute $12[g]$ de crème à $4\celsius$ dont la chaleur massique est de $3350[\cm]$. On rajoute une tasse en pyrex dont la chaleur massiques est de $830[\cm]$. On prend en compte l'air dont la chaleur massique est de $1000[\cm]$ et la masse volumique de $1.28[\frac{kg}{m^3}]$.Le volume de l'air est égal à $0.025^2 * 0.02 * \pi = 3.927*10^{-5}[m^3]$.

\textit{A quelle température allez-vous boire le café ?}
\subsubsection{Valeurs}
\begin{itemize}
    \item[crème] (1)
    \item[] 
    \begin{itemize}
        \item[$T1$] = $4\celsius$
        \item[$m1$] = $12[g]$ = $0.012[kg]$
        \item[$Cm1$] = $3350[\cm]$
    \end{itemize} 
    \item[café] (2)
    \item[] 
    \begin{itemize}
        \item[$T2$] = $60\celsius$
        \item[$m2$] = $1[dl]$ = $0.1[kg]$
        \item[$Cm2$] = $4180[\cm]$   
    \end{itemize} 
    \item[tasse en pyrex] (3)
    \item[] 
    \begin{itemize}
        \item[$T3$] = $20\celsius$
        \item[$m3$] = $0.1[kg]$
        \item[$Cm3$] = $830[\cm]$   
    \end{itemize} 
    \item[air] (4)
    \item[] 
    \begin{itemize}
        \item[$T4$] = $20\celsius$
        \item[${\rho}4$] = $1.28[\frac{kg}{m^3}]$
        \item[$V4$] = $3.927*10^{-5} [m^3]$ 
        \item[$m4$] = $mV * V = 1.28 * 3.927*10^{-5} = 5.027*10^{-5}[kg]$
        \item[$Cm4$] = $1000[\cm]$   
    \end{itemize} 
\end{itemize}
\subsubsection{Reponse}
\begin{itemize}
    \item[Calcul des Q] : $Q = Eth = m * Cm * (Tf-Ti)$
    \item $Q1$ = $0.012 * 3350 * (Tf - 4) = 40.2 * (Tf-4) = 40.2Tf - 160.80$
    \item $Q2$ = $0.1 * 4180 * (Tf - 60)$ = $418 * (Tf - 60)$ = $418Tf - 25080$
    \item $Q3$ = $0.1 * 830 * (Tf - 20)$ = $83 * (Tf - 20)$ = $83Tf - 1660$
    \item $Q4$ = $(5.027*10^{-5}) * 1000 * (Tf - 20) = (5.027*10^{-2}) * (Tf-20) = (5.027*10^{-2})Tf - 1.005$
    \item[Conservation des Q] : $Q1 + Q2 +Q3 + Q4 = 0$
    \item $Q1 + Q2 + Q3 + Q4 = 541.25Tf - 26579.2 = 0$
    \item $Tf$ = $\frac{26579.2}{541.25} = 49.11\celsius $
\end{itemize}
