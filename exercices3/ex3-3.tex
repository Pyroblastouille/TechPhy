\subsection{Exercice 3}
\subsubsection{Enoncé}
Une tasse de café à $60\celsius$ de 1dl dont la chaleur massique est de $4180[\cm]$. On y rajoute $12[g]$ de crème à $4\celsius$ dont la chaleur massique est de $3350[\cm]$. On rajoute une tasse en pyrex dont la chaleur massiques est de $830[\cm]$. On prend en compte l'air dont la chaleur massique est de $1000[\cm]$ et la masse volumique de $1.28[\kgmv]$.Le café est à $30\celsius$ après.


\textit{Quel était le volume d'air autour du café ?}
\subsubsection{Valeurs}
\begin{itemize}
    \item[crème] (1)
    \item[] 
    \begin{itemize}
        \item[$T1$] = $4\celsius$
        \item[$m1$] = $12[g]$ = $0.012[kg]$
        \item[$Cm1$] = $3350[\cm]$
    \end{itemize} 
    \item[café] (2)
    \item[] 
    \begin{itemize}
        \item[$T2$] = $60\celsius$
        \item[$m2$] = $1[dl]$ = $0.1[kg]$
        \item[$Cm2$] = $4180[\cm]$   
    \end{itemize} 
    \item[tasse en pyrex] (3)
    \item[] 
    \begin{itemize}
        \item[$T3$] = $20\celsius$
        \item[$m3$] = $0.1[kg]$
        \item[$Cm3$] = $830[\cm]$   
    \end{itemize} 
    \item[air] (4)
    \item[] 
    \begin{itemize}
        \item[$T4$] = $20\celsius$
        \item[${\rho}4$] = $1.28[\frac{kg}{m^3}]$
        \item[$V4$] = $3.927*10^{-5} [m^3]$ 
        \item[$m4$] = $?$
        \item[$Cm4$] = $1000[\cm]$   
    \end{itemize} 
\end{itemize}
\subsubsection{Reponse}
\begin{itemize}
    \item[Calcul des Q] : $Q = Eth = m * Cm * (Tf-Ti)$
    \item $Q1$ = $0.012 * 3350 * (30 - 4) = 1045.2[J] $
    \item $Q2$ = $0.1 * 4180 * (30 - 60) = -12540[J]$
    \item $Q3$ = $0.1 * 830 * (30 - 20) = 830[J]$
    \item $Q4$ = $(?) * 1000 * (30 - 20) = (?) * 10^{4}$
    \item[Conservation des Q] : $Q1 + Q2 +Q3 + Q4 = 0$
    \item $Q1 + Q2 + Q3 + Q4 = -10664.8 + (?) * 10^4 = 0$
    \item $(?) = \frac{10664.8}{10^4} = 1.06648[kg]$ 
    \item $V4 = \frac{1.06648}{{\rho}4} = 0.833[m^3]$
\end{itemize}
