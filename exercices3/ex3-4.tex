\subsection{Exercice 4}
\subsubsection{Enoncé}
Un thermos dont l'exterieur est à $20\celsius$ dans lequel on met $1litre$ d'eau à $90\celsius$. Après beaucoup de temps, on vérifie la température de l'eau qui est arrivé à $89\celsius$.

\textit{Quel est la Chaleur Spécifique du thermos?}
\subsubsection{Valeurs}
\begin{itemize}
    \item[eau] (1)
    \item[] 
    \begin{itemize}
        \item[$Ti1$] = $90\celsius$
         \item[$Tf1$] = $89\celsius$
        \item[${\rho}1$] = $998[\kgmv]$
        \item[$V1$] = $1[l]=1[dm^3] = 0.001[m^3]$
        \item[$m1$] = $0.998[kg]$
        \item[$Cm1$] = $4180[\cm]$
    \end{itemize} 
    \item[thermos] (2)
    \item[] 
    \begin{itemize}
        \item[$Ti2$] = $20\celsius$
         \item[$Tf2$] = $80\celsius$
        \item[$Cs2$] = $?$   
    \end{itemize} 
\end{itemize}
\subsubsection{Reponse}
\begin{itemize}
    \item[Calcul du Q de l'eau] :  $Q = Cm * m * (Tf- Ti)$
    \item $Q1 = 4180 * 0.998 * (89-90) = -4171.64[J]$
    \item[Calcul du Q du thermos] :  $Q = Cs * (Tf-Ti)$
    \item $Q2 = ? * (89-20)$
    \item[Conservation de Q] : $Q1 + Q2 = 0$
    \item $Q2 = -Q1$
    \item $? * 69 = -(-4171.64)$
    \item $? = 4171.64 / 69 = 60.459[\cs]$
\end{itemize}
La chaleur spécifique du thermos est de $60.459[\cs]]$.
