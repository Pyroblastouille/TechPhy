\subsection{Exercice 1-1}
\subsubsection{Enoncé}
Un bol d'eau de $1 litre$ dont la température est de $14\celsius$. La chaleur massique de l'eau est de $4180[\cm]$.
\textit{Quel est la chaleur nécessaire pour l'amener à $90\celsius$ ?}
\subsubsection{Valeurs}
\begin{itemize}
    \item $m$ = $1 litre \Rightarrow 1[kg]$
    \item $Cm$ = $4180[\cm]$
    \item $Tf$ = $90\celsius$
    \item $Ti$ = $14\celsius$
\end{itemize}
\subsubsection{Reponse}
\begin{itemize}
    \item $Q = Eth = m * Cm * (Tf-Ti) = 1*4180*(90-14) = 434720[J]$
\end{itemize}

\subsection{Exercice 1-2}
\subsubsection{Enoncé}
 Le lac de Genève contient $89[km^3]$.La chaleur massique de l'eau est de $4180[\cm]$.
\textit{Quel est la chaleur nécessaire pour augmenter la température de $2\celsius$ ?}
\subsubsection{Valeurs}
\begin{itemize}
    \item $m$ = $89km^3 = 89*10^{12} litres \Rightarrow 89*10^{12}[kg]$
    \item $Cm$ = $4180[\cm]$
    \item ${\Delta}T$ = $2\celsius$
\end{itemize}
\subsubsection{Reponse}
\begin{itemize}
    \item $Q = Eth = m * Cm * {\Delta}T = 89*10^{12}*4180*2 = 7.4404*10^{17}[J]$
\end{itemize}
