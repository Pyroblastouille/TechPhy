\subsection{Exercice 5}
\subsubsection{Enoncé}


\textit{Quel est la Chaleur massique du machin qu'on sait pas?}
\subsubsection{Valeurs}
\begin{itemize}
    \item[eau] (1)
    \item[] 
    \begin{itemize}
        \item[$Cm$] = $4180[\cm]$
        \item[$m$] = $400[g] ou 0.4[kg]$
        \item[$Ti$] = $20\celsius$
        \item[$Tf$] = $21.4\celsius$  
    \end{itemize}
    \item[onsaispas] (2)
    \item[] 
    \begin{itemize}
        \item[$Cm$] = $?[\cm]$
        \item[$m$] = $50[g] ou 0.05[kg]$
        \item[$Ti$] = $400\celsius$
        \item[$Tf$] = $21.4\celsius$  
    \end{itemize}
    \item[calorimètre] (3)
    \item[] 
    \begin{itemize}
        \item[$Cs$] = $80[\cs]$
        \item[$Ti$] = $20\celsius$
        \item[$Tf$] = $21.4\celsius$
    \end{itemize} 
\end{itemize}
\subsubsection{Reponse}
\begin{itemize}
    \item[Calcul des Q] : $Q = Cm * m * (Tf-Ti)$
    \item $Q1 = 4180 * 0.4 * (21.4 - 20) = 2340.8[J]$
    \item $Q2 = Cm2 * 0.05 * (21.4 - 400)$
    \item $Q3 = Cs3 * (Tf-Ti3) = 80 * (21.4 - 20) = 112[J]$
    \item[Conservation des Q] : $Q1 + Q2 + Q3 = 0$
    \item $2340.8 + Q2 + 112 = 0$
    \item $Q2 = -2452.8$
    \item $Cm2 = \frac{-2452.8}{0.05 * (21.4 - 400)} = 129.5721078[\cm]$
\end{itemize}
La chaleur massique du matériau est de $129.5721078[\cm]$.
