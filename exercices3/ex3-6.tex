
\subsection{Exercice 6}
\subsubsection{Enoncé}

Un glaçon de $10[g]$ se retrouve en une flaque après plusieurs heures.

\textit{Quel est l'energie absorbée ?}
\subsubsection{Valeurs}
\begin{itemize}
    \item[glace] (1)
    \item[] 
    \begin{itemize}
        \item[$Cm$] = $2060[\cm]$
        \item[$m$] = $10[g] ou 0.01[kg]$
        \item[$Ti$] = $-18\celsius$
        \item[$Tf$] = $0\celsius$  
    \end{itemize}
    \item[fusion] (2)
    \item[] 
    \begin{itemize}
        \item[$m$] = $10[g] ou 0.01[kg]$
        \item[$Lf$] = $3.3 * 10^5[\lf]$
    \end{itemize} 
    \item[eau] (3)
    \item[] 
    \begin{itemize}
        \item[$Cm$] = $4180[\cm]$
        \item[$m$] = $10[g] ou 0.01[kg]$
        \item[$Ti$] = $0\celsius$
        \item[$Tf$] = $21\celsius$  
    \end{itemize}
\end{itemize}
\subsubsection{Reponse}
\begin{itemize}
    \item[Calcul de Q1 et Q3] : $Q = Cm * m * (Tf-Ti)$
    \item $Q1 = 2060 * 0.01 * (0-(-18)) = 370.8[J]$
    \item $Q3 = 4180 * 0.01 * (21-0) = 877.8[J]$
    \item[Calcul de Q2] : $Q = m * Lf$
    \item $Q2 = 0.01 * 3.3*10^5 = 3300[J]$  
    \item[Calcul du Qtotal] : $Qtot = Q1+ Q2+Q3$
    \item $Qtot = 370.8 + 877.8 + 3300 = 4548.6 [J]$  
\end{itemize}
Le glaçon à développer $4548.6[J]$ pour se retrouver en flaque.
La fusion comprend 72.5\% de cette energie.
