La quantité de mouvement est égale à la vitesse multipliée par la masse
$_{P}^{\rightarrow} = m * _{V}^{\rightarrow}$.
Son signe est alors $[kg*\frac{m}{s}]$.
\\
La somme des quantité de mouvement avant sera toujours égale à la somme des quantité de mouvement après

$\sum_{i=1}^{n}$ $_{P}^{\rightarrow}avant = $$\sum_{i=1}^{m}$ $_{P}^{\rightarrow}apres$
\\
Qui peut être écrit : 

$m1 * (_{V1y}^{V1x}) + m2 * (_{V2y}^{V2x}) = m3 * (_{V3y}^{V3x}) + ...$
