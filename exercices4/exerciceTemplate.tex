
\subsection{Exercice 2}
\subsubsection{Enoncé}
La pièce pert $17821[J]$ avec une température de $20\celsius$ au départ. Quel est sa temperature sachant que la pièce fait $10m$ par $10m$ sur une hauteur de $3m$ et que la masse volumique de l'air est de $1.225[\mv]$. La chaleur massique de l'air est de $1000[\cm]$.
\textit{Quel est la temperature finale de la pièce ?}
\subsubsection{Valeurs}
\begin{itemize}
    \item[Air] (1)
    \item[] 
    \begin{itemize}
        \item[$Cm$] = $4180[\cm]$
    \end{itemize}
\end{itemize}
\subsubsection{Reponse}
\begin{itemize}
    \item[Calcul de Q1 et Q3] : $Q = Cm * m * (Tf-Ti)$
\end{itemize}