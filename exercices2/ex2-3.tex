\subsection{Exercice 3}
\subsubsection{Enoncé}
La quantité de mouvement nous permet également de trouver la masse d'un objet.

On prend un objet $B$ dont on connait la masse ($500[g]$) et la vitesse ($(_{0}^{0})[\frac{m}{s}]$)

Nous avons un objet A dont on connait sa vitesse ($(_{0}^{4})[\frac{m}{s}]$) mais pas sa masse



Après le choc, l'objet B est repoussé à une vitesse égale à $_{V'B}^{\rightarrow} = (_{0.00}^{0.54})[\frac{m}{s}]$ alors que l'objet A part dans l'autre direction à une vitesse de $2.48[\frac{m}{s}]$

\subsubsection{Valeurs}
\begin{itemize}
   \item[$_{V_B}^{\rightarrow}$] $ = (_{0}^{0})[\frac{m}{s}]$
   \item[$m_B$] $ = 500[g]$
   \item[$_{V_A}^{\rightarrow}$] $ = (_{0}^{4})[\frac{m}{s}]$
   \item[$m_A$] $ = ?$
   \item[$_{V'_A}^{\rightarrow}$] $ = (_{+0.00}^{-2.48})[\frac{m}{s}]$
   \item[$_{V'_A}^{\rightarrow}$] $ = (_{0.00}^{0.54})[\frac{m}{s}]$
   
\end{itemize}

\subsubsection{Réponse}
\begin{itemize}
    \item $_{P_A}^{\rightarrow} = (_{0}^{4}) * m_A$
    \item  $_{P_B}^{\rightarrow} = (_{0}^{0}) * 0.5 = (_{0}^{0})$
    \item  $_{P'_B}^{\rightarrow} = (_{0.00}^{0.54}) * 0.5 = (_{0.00}^{0.27})$
    \item  $_{P'_A}^{\rightarrow} = (_{+0.00}^{-2.48}) * m_A$
    \item $_{P_A}^{\rightarrow} + _{P_B}^{\rightarrow} = _{P'_A}^{\rightarrow} + _{P'_A}^{\rightarrow}$ car conservation de $_{P}^{\rightarrow}$
    \item $(_{0}^{4}) * m_A + (_{0}^{0}) = (_{+0.00}^{-2.48}) * m_A + (_{0.00}^{0.27})$
    \item $(_{0}^{4}) * m_A - (_{+0.00}^{-2.48}) * m_A = (_{0.00}^{0.27})$
    \item $(_{0.00}^{6.48}) * m_A = (_{0.00}^{0.27})$
    \item $m_A = 0.041666667[kg] \Rightarrow 42[g]$
\end{itemize}
