\subsection{Exercice 4}
\subsubsection{Enoncé}
Une météorite de $1[t]$ dont la vitesse est de $(108[\frac{km}{h}];0^\circ)$ rentre en collision avec une autre météorite de $4[t]$ dont la vitesse est de $(36[\frac{km}{h}];50^\circ)$.
Après la collision, la deuxième météorite part avec une vitesse de $(8[\frac{m}{s}];25^\circ)$

\textit{Quel est la vitesse $_{V}^\rightarrow$ de la première météorite après la collision?}

\subsubsection{Valeurs}
\begin{itemize}
    \item[\textbf{$M1$}]
    \item $m1$ = $1[t] \Rightarrow 1000[kg]$
    \item $_{V1}^{\rightarrow}$ = $ (108[\frac{km}{h}];0^\circ) \Rightarrow (30[\frac{m}{s}];0^\circ) ou (^{30*\cos(0)}_{30*\sin(0)})=(^{30}_{00})[\frac{m}{s}]$
    \item $_{V1'}^{\rightarrow}$ = ?
    \item[\textbf{$M2$}]
    \item $m2$ = $4[t] \Rightarrow 4000[kg]$
    \item $_{V2}^{\rightarrow}$ = $ (36[\frac{km}{h}];50^\circ) \Rightarrow (10[\frac{m}{s}];50^\circ) ou (^{6.428}_{7.660})[\frac{m}{s}]$ 
    \item $_{V2'}^{\rightarrow}$ = $(8[\frac{m}{s}];25^\circ) ou (^{7.250}_{3.381})[\frac{m}{s}]$
\end{itemize}

\subsubsection{Réponse}
Calculer les quantités de mouvement
\begin{itemize}
    \item $_{P1}^{\rightarrow}$ = $_{V1}^{\rightarrow} * m1 = (^{30}_{00}) * 1000 = ({^{30000}_{0}})[kg*\frac{m}{s}]$
    \item $_{P2}^{\rightarrow}$ =  $_{V2}^{\rightarrow} * m2 = (^{6.428}_{7.660}) * 4000 = ({^{25712}_{30640}})[kg*\frac{m}{s}]$
    
    \item $_{P1'}^{\rightarrow}$ = $_{V1'}^{\rightarrow} * m1 = ? * 1000$  
    \item $_{P2'}^{\rightarrow}$ =  $_{V2'}^{\rightarrow} * m2 = (^{7.250}_{3.381}) * 4000 = ({^{29000}_{13524}})[kg*\frac{m}{s}]$
    \item[$_{P1}^{\rightarrow} + _{P2}^{\rightarrow} = _{P1'}^{\rightarrow} + _{P2'}^{\rightarrow}$] car conservation de la quantité de mouvement
    \item $_{P1'}^{\rightarrow}$ = $_{P1}^{\rightarrow}+_{P2}^{\rightarrow} - _{P2'}^{\rightarrow} =({^{30000}_{00000}})+ ({^{25712}_{30640}}) - ({^{29000}_{13524}}) = ({^{26712}_{17116}}) [kg*\frac{m}{s}]$
    \item $_{V1'}^{\rightarrow}$ = $_{P1'}^{\rightarrow} / m1 = ({^{26712}_{17116}})) / 1000 = ({^{26.712}_{17.116}})[\frac{m}{s}] \Rightarrow (31.725[\frac{m}{s}];32.65^\circ) \Rightarrow (114.21[\frac{km}{h}];32.65^\circ)$
\end{itemize}
   
La vitesse de la première météorite après impact est égale à 

$(31.725[\frac{m}{s}];32.65^\circ)$.
