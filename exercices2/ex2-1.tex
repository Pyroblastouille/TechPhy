\subsection{Exercice 1}
\subsubsection{Enoncé}
Une voiture de 0.9 tonnes qui fait la course avec une vitesse de 162 km/h rentre en collision avec une autre voiture de 1,4 tonnes venant de sa gauche à angle droit et qui roule normalement à 14m/s.
Les 2 voitures reste ensemble après le choc

\textit{Trouvez la $_{V}^\rightarrow$ après le choc ?}

\subsubsection{Réponse}
$_{V1}^\rightarrow = (_{00}^{45})[\frac{m}{s}]$

$_{V2}^\rightarrow = (_{-14}^{+00})[\frac{m}{s}]$


$_{P1}^\rightarrow = 900 * (_{00}^{45}) = (_{00000}^{40500})[kg*\frac{m}{s}]$

$_{P2}^\rightarrow = 1400 * (_{-14}^{+00}) = (_{-19600}^{+00000})[kg*\frac{m}{s}]$

$_{P1}^\rightarrow + _{P2}^\rightarrow = _{P3}^\rightarrow = (_{-19600}^{+40500})[kg*\frac{m}{s}]$

$_{V3}^\rightarrow =  _{P3}^\rightarrow / (m1+m2) =  (_{-19600}^{+40500}) / (900+1400) = (_{-08.52}^{+17.61})[\frac{m}{s}]$

$_{V3}^\rightarrow = (_{-08.52}^{+17.61})[\frac{m}{s}] = (19.56[\frac{m}{s}] ; 334.18^{\circ} ) \approx (70[\frac{km}{h}];334.18^{\circ} )$
