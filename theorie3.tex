\subsection{Température vs Chaleur}
La température c'est l'énergie moyenne des particules exprimée en degrès.
A 0 degrès Kelvin les particules sont figées.

Signe international = $Kelvin (K)$ 

$0[K]$ = $-273.15^{\circ}C$


La chaleur est l'énergie fournie pour augmenter ou diminuer la température.
L'objet dont la température baisse expulse de la chaleur alors que si on donne de la chaleur, la température augmente.

Signe international = $[J]$

\subsection{Formules}
\begin{itemize}
    \item[Chaleur] $Q = Eth=m*Cm*(Tf-Ti)$ 
    \item $m$ = masse $[kg]$
    \item $Cm$ = Chaleur massique $[\cm]$
    \item $Tf-Ti$ = Temperature ($K$ ou $^{\circ}C$)
    \item[Complexe] $Q = Cs * (Tf - Ti)$
    \item $Cs$ = Chaleur Spécifique $[\cs]$
    \item $Tf-Ti$ = Temperature ($K$ ou $^{\circ}C$)
\end{itemize}

\subsection{Changement d'état}
$Q = m * Lf$
où $Lf$ = Chaleur latente de fusion = $[\lf]$
\subsection{Transformation physique vs chimique}
Une transformation physique est reversible alors qu'une transformation chimique ne l'est pas.
