\subsection{Conduction}
Transfert par le plus proche voisin principalement dans les solides.

La Chaleur $Q$ transmise pendant une durée $\dt$ au travers d'une surface $S$ dépend de:
\begin{itemize}
    \item[Historique :] $Q = \lambda * \frac{S}{e} *(T1-T2)* \dt = [J]$
    \item $S [m^2] =$ une surface
    \item $e [m] =$ une épaisseur
    \item $\lambda [\lbda] =$ la conductivité thermique du matériau
    \item $\dt [s] =$ la difference de temps
    \item $T1 [K]=$ Temperature environnement 1
    \item $T2 [K]=$ Temperature environnement 2
    \item[Resistance thermique]
    \item$Rtot = \frac{e1}{{\lambda}1}+\frac{e2}{{\lambda}2}+\frac{e3}{{\lambda}3}$
    \item$Rtot = \Sigma^{n}_{i=1} = \frac{ei}{{\lambda}i} [\rtot]$
    \item$Q = \frac{1}{Rtot} * S *(T1-T2) * \dt$ 
\end{itemize}



\subsection{Convection}
Transfert par déplacement de matière dans les fluides (liquides-gaz).

\begin{itemize}
    \item $Q = S * \alpha * (T1-T2) * \dt$
    \item $P = \frac{Q}{\dt} [W]$
    \item[Valeurs]
    \item $S =$ une surface
    \item $\alpha [\conv]=$ coefficient de convection
    \item $T1 [K]=$ Temperature environnement 1
    \item $T2 [K]=$ Temperature environnement 2
    \item $\dt [s] =$ la difference de temps
\end{itemize}

\subsection{Rayonnement}
Transfert par onde électromagnétique. 

\begin{itemize}
    \item Wololoooo
\end{itemize}