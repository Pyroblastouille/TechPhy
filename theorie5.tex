
Conséquence d'un changement de température
Par exemple, les rails étaient pas collées pour éviter les problèmes liés à la dilatation. On met maintenant des éléments ductiles (qui peut être allongé sans se rompre)

Sans espace, les rails vont "flamber" c'est à dire faire une courbure et ça risque de se fissurer et tout casser

\subsection{Solides}
Si on prend un rail et qu'on augmente la température, la longueur augmente de chaque côté de $\frac{{\Delta}l}{2}$. Si un coté est bloqué, l'autre coté augmente de  ${\Delta}l$.

\begin{itemize}
    \item[${\Delta}l = \alpha * l1 * (T2 - T1)$]
    \item ${\Delta}l = [m]$
    \item $\alpha = \frac{1}{K}$ le Coefficient de dilatation linéaire
    \item $l1 = [m]$
    \item $(T2 - T1) = K ou \celsius$
    \item[$l2 = l1 + {\Delta}l$]
    \item[$l2 = l1 + \alpha * l1 * (T2 - T1)$]
    \item[$l2 = l1 (1+\alpha*(T2-T1)$]
\end{itemize}


\subsection{Liquides}
\subsection{Gazeux}
