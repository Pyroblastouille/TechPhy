
Conséquence d'un changement de température
Par exemple, les rails étaient pas collées pour éviter les problèmes liés à la dilatation. On met maintenant des éléments ductiles (qui peut être allongé sans se rompre)

Sans espace, les rails vont "flamber" c'est à dire faire une courbure et ça risque de se fissurer et tout casser

\subsection{Solides}
Si on prend un rail et qu'on augmente la température, la longueur augmente de chaque côté de $\frac{{\Delta}l}{2}$. Si un coté est bloqué, l'autre coté augmente de  ${\Delta}l$.

\begin{itemize}
    \item[$\dl = \alpha * l1 * (T2 - T1)$]
    \item $\dl = [m]$
    \item $\alpha [\dila]= $ le Coefficient de dilatation linéaire
    \item $l1 = [m]$
    \item $(T2 - T1) = K ou \celsius$
    \item[$l2 = l1 + \dl$]
    \item[$l2 = l1 + \alpha * l1 * (T2 - T1)$]
    \item[$l2 = l1 (1+\alpha*(T2-T1)$]
\end{itemize}

En 3D il faut faire la même chose dans chaque direction

\subsection{Liquides}
Il faut prendre en compte le récipient qui est solide.Dans les liquides ça dilate beaucoup plus. On va faire directement de la dilatation volumique

\begin{itemize}
    \item $V2 = V1 + \dV$
    \item $\dV = \gamma * V1 * \dT$
    \item $\gamma [\dila] =$ = Coefficient de dilatation volumique
\end{itemize}

\subsection{Gazs}
Ils se dilatent presque tous de la même manière. (Si à la même pression)
$\gamma \frac{1}{273} [\dila]$

\begin{itemize}
    \item[Loi des gazs parfaits] $\frac{P*V}{T} = constante$
    \item $P [Pa] ou [bar]$
    \item $V volume [m^3] ou [l]$
    \item $T = [K]$
\end{itemize}