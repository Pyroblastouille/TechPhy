
\subsection{Exercice 1}
\subsubsection{Enoncé}
Calculez la variation max de la hauteur du Burj Khalifa.
\subsubsection{Valeurs}
\begin{itemize}
    \item[Burj Khalifa]
    \item hauteur à $30\celsius = 828[m]$
    \item $Tmin = 10\celsius$
    \item $Tmax = 48\celsius$
    \item ${\alpha}beton = 10^{-5}[\frac{1}{K}]$
\end{itemize}
\subsubsection{Reponse}
\begin{itemize}
    \item[Calcul de ${\Delta}l$ à $Tmin$] ${\Delta}l = \alpha * l * (T2-T1)$
    \item ${\Delta}l = 10^{-5} * 828 * (10-30) = -0.1656[m]$
    \item[Calcul de ${\Delta}l2$ à $Tmax$] ${\Delta}l2 = \alpha * l * (T2-T1)$
    \item ${\Delta}l2 = 10^{-5} * 828 * (48-30) = 0.14904[m]$
    \item[Calcul de différence max] ${\Delta}l2 - {\Delta}l = {\Delta}Max$
    \item ${\Delta}Max = 0.14904 - (-0.1656) = 0.31464 [m]$
\end{itemize}
