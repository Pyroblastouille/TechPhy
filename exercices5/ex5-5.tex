
\subsection{Exercice 5}
\subsubsection{Enoncé}
On va faire des thermomètres:
\begin{itemize}
    \item 1 avec du mercure
    \item 1 avec de l'alcool
\end{itemize}
On veut 1 $[mm]$ pour $1\celsius$.

\textit{Quel volume de mercure doit contenir le tube en $[mm^3]$ pour fonctionner correctement si on néglige la dilatation du tube en verre?}
\textit{Si on remplace le mercure par l'alcool et que l'on garde le même volume, calculez la distance entre 2 graduations en $[mm]$}
\subsubsection{Valeurs}
\begin{itemize}
    \item $\Delta{T} = 1\celsius$
    \item $\Delta{Volume} = 0.0314[mm^{3}]$
    \item[Tube en verre]
    \item $diametre 0,2[mm]$
    \item[mercure]
    \item $\gamma{Mercure} = 1.72*10^{-4}[\dila]$
    \item[alcool] 
    \item $\gamma{Alcool} = 1.10*10^{-3}[\dila]$
\end{itemize}

\subsubsection{Réponses}
\begin{itemize} 
    \item[Volume de mercure]
    \item $V1 = \frac{\dV}{\gamma * \dT}$
    \item $0,0314 = V1 * 1.72*10^{-4} * 1$
    \item $V1 = \frac{0,0314}{1.72*10^{-4}} = 182.558[mm^3] = 0.182558[ml]$
    \item[$\dV$ alcool]
    \item $\dV = V1*\gamma * \dT$
    \item $\dV = 182.558 * 1.10*10^{-3} * 1 = 0.2008138[mm^3]$
    \item $hauteurTube = 0.2008138 / (r^2 * \pi) = 6.392101782 [mm]$ 
\end{itemize}

Volume du mercure = $0.183[ml]$

Graduation par degré avec l'alcool = $6.39[mm]$
