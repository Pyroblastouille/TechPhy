
\subsection{Exercice 3}
\subsubsection{Enoncé}
Il faut faire rentrer une barre de cuivre dans un emplacement d'une pièce d'acier en refroidissant le cuivre avec de l'azote liquide. On veut une tolérance de $\frac{1}{10}mm$ pour le cuivre à la température de l'azote liquide ).
\textit{Calculez la longueur de la piece en cuivre à $20\celsius$}

\subsubsection{Valeurs}
\begin{itemize}
    \item[Cuivre]
    \item ${\alpha}cuivre = 16.6 * 10^{-6}[\frac{1}{K}]$
    \item $l2 = 11.99[cm] = 0.1199 [m]$
    \item $l1 = ? [m]$
    \item $T2 = 20\celsius$
    \item[Acier]
    \item ${\alpha}acier = 11 * 10^{-6}[\frac{1}{K}]$
    \item[Azote liquide]
    \item  $Tazote = -196\celsius$
\end{itemize}

\subsubsection{Réponses}
\begin{itemize} 
    \item $\dl = \alpha * l * (T2 - Tazote)$
    \item $\dl = 16.6 * 10^{-6} * 0.1199 * (20 - (-196)) = 4.299 * 10^{-4} [m] \approx 0.04[cm]$
    \item $l1 = 0.1203299[m] \approx 12.03 [cm]$
    \item[Autre technique]
    \item $l2 = l1 + \dl$
    \item $11.99 = l1 + 16.6*10^{-6}*l1*(-196-20)$
    \item $11.99 = l1 + l1 * (16.6*10^{-6}*-216) = l1+l1 * (-3.58*10^{-3})$
    \item $11.99 = l1(1 + (-3.58*10^{-3})) = l1 * (0.9964144)$
    \item $l1 = \frac{11.99}{0.9964144} = 12.033[cm]$ 
\end{itemize}
