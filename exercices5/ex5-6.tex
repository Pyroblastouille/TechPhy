
\subsection{Exercice 6}
\subsubsection{Enoncé}
Trouver la pression dans l'ampoule.
\subsubsection{Valeurs}
\begin{itemize}
    \item[Ampoule]
    \item $T1 = 20\celsius = 293.15 [K]$
    \item $T2 = 100\celsius = 373.15 [K]$  
    \item $P1 = \frac{1}{2}[bar]$
    \item $P2 = ?$
\end{itemize}

\subsubsection{Réponses}
\begin{itemize} 
    \item[Calcul P2 ] $\frac{P1*V1}{T1} = \frac{P2*V2}{T2}$
    \item $\frac{0.5 * 1}{293.15} = \frac{P2 * 1}{373.15}$
    \item $P2 = \frac{0.5*373.15}{293.15} = 0.636 [bar]$  
\end{itemize}

Volume du mercure = $0.183[ml]$
Volume du mercure = $0.183[ml]$

Graduation par degré avec l'alcool = $6.39[mm]$
